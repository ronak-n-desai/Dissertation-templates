\chapter{Introduction}

%\begin{quote}
%\lipsum[1]
%\end{quote}
\section{Lasers}

Explain what a laser stands for, stimulated emission, lasing medium, what the Ti:sapphire crystal does and why it is chosen, CPA

\subsection{Chirped Pulse Amplification}

\subsection{...}

%\section{Glossaries}
%Glossaries are disabled in this version of the template - if you are interested get osudiss-template\_glossary.
%The latex glossaries package can be useful for keeping track of your acronyms, and making a nice hyperlinked 
%list at the beginning of your document.  Note: a glossary/list of acronyms is NOT required by the GS.
%If you do wish to include one, it \textbf{must} appear directly after the lists of figures and tables.
%To use the glossaries package, you must load it in your preamble:
%\begin{verbatim}
%\usepackage[acronym, section=chapter]{glossaries} %load package
%\makeglossaries %required to actually make a glossary
%\include{acronyms} %load list of acronyms contained in acronyms.tex
%\end{verbatim}
%Then, to define an acronym, you will need to include it in your preamble:
%\begin{verbatim}
%\newacronym{AFM}{AFM}{atomic force microscopy}
%\end{verbatim}
%In this template, the acronyms are all placed in a separate file called
%\verb#acronmys.tex# which is then loaded with \verb#\include{acronyms}#, to
%make the main source file easier to read.  You can also just put all
%your acronyms directly in the header of your main source file.
%
%Once you have defined an acronym, you can use it with the \verb#\gls{<label>}# command.
%The first time you use the acronym, the full definition is printed: atomic force microscopy (AFM). %\gls{AFM}.
%On subsequent uses, just the abbreviation is printed: AFM %\gls{AFM}
% - Latex keeps track
%for you, so you don't have to do this manually.  There are fancier forms
%of \verb#\gls# that allow you to capitalize (\verb#\Gls{<label>}#) or use the 
%plural (\verb#\glspl{<label>}#) forms of your abbreviations.  See 
%for example \url{http://en.wikibooks.org/wiki/LaTeX/Glossary} for details
%on the glossaries package.

%If you want to list all your acronyms at the beginning of your document, you
%will need to include the \verb#\makeglossaries# command in your preable, as
%shown above, and then 
%\begin{verbatim}
%\printglossary[type=\acronymtype]
%\end{verbatim}
%where you want the list of abbreviations to actually appear 
%(as stated above, this must be right after your lists of figures and tables in the roman
%numeral pages at the beginning of the document).  Then you have
%to go to the terminal (in the directory containing your document), and run the command
%\begin{verbatim}
%makeglossaries (yourdocumentname)
%\end{verbatim}
%to actually create the glossary.  If you just want Latex to keep track of your acronyms for you 
%and print no list, you can just skip this step and the \verb#printglossary# command.

%To make the heading for the List of Abbreviations look the same as the List of Figures
%and List of Tables, instead of using \verb#\printglossary# I defined a custom command to print the glossary:
%\begin{verbatim}
%\newcommand\PrintListofAbbreviations[1]{
%\phantomsection
%\addcontentsline{toc}{front}{\typesetColumnHeading{#1}}
%\printglossary[type=\acronymtype,title={\protect {\typesetLevelTwo{#1}}}]
%%\end{verbatim}
%This command functions the same as \verb#\printglossary#, but instead
%of typesetting the heading like a chapter heading (in smallcaps), 
%it typesets in bold like the lists of figures and tables.  
%To use it, just call \verb#\PrintListofAbbreviations{List of Abbreviations}#
%right after \verb#\listoftables# and before \verb#\mainmatter#.
%If you wish to use a different title for you list of abbreviations,
%just call \verb#\PrintListofAbbreviations{Your Title}#.
%I am unsure if the GS has any restrictions on what title you can use.
%
%One final comment, this template uses \verb#makeindex# to create the glossaries
%for compatability (some versions of Ubuntu for example don't support \verb#xindy#).
%However, there is a better program for making glossaries that you can use by calling
%\begin{verbatim}
%\usepackage[xindy,toc,acronym, section=chapter]{glossaries}
%\end{verbatim}
%instead.  In particular, \verb#xindy# allows you to use symbols in your abbreviations,
%such as Greek letters or accented characters, which are not supported by \verb#makeindex#.
%
%This figure also illustrates the use of subfigures via the subfig package, loaded by placing 
%\verb#\usepackage{subfig}# in the preamble of \verb#template.txt#.  This allows you to include
%each subfigure as an individual image, and have latex arrange and label them for you 
%(including an optional name for each subfigure in \verb#[]#).  This also allows
%you to individually reference Fig.~\ref{one_battery} and Fig.~\ref{five_battery}.
%Note that many journals DO NOT support this, and require that each figure be a single
%image, but this should be fine for a thesis.  
%
%A footnote\footnote{Another foot.}

\section{High Energy Density Science}

\subsection{Fusion}


\subsection{Particle Acceleration}


\section{This Work}

Talk about prior work 


Talk about motivating question 


Talk about how work is organized. 


